\documentclass[a4paper]{article} 

% deutsche Sonderzeichen benutzen
%\usepackage[german]{babel}

% wegen deutschen Umlauten
\usepackage[utf8]{inputenc}

\usepackage[T1]{fontenc}
% hat was mit Abstaenden zu tun
\frenchspacing

\usepackage{moreverb}
%urls with hyperref
\usepackage{hyperref}

\usepackage{listings}

\usepackage{tikz}
%\tikzstyle{every picure}+=[remember picture]
%\tikzstyle{na} = [baseline=-.5ex]


%urls with url
\usepackage{url}
%% Define a new 'leo' style for the package that will use a smaller font.
\makeatletter
\def\url@leostyle{%
  \@ifundefined{selectfont}{\def\UrlFont{\sf}}{\def\UrlFont{\small\ttfamily}}}
\makeatother
%% Now actually use the newly defined style.
\urlstyle{leo}

%multiple rows in table
\usepackage{multirow}
\usepackage{latexsym}
% absolut positioning of text with
% env: \begin{textblock}{<size>}( hpos , vpos )

\usepackage[absolute]{textpos}
\setlength{\TPHorizModule}{30mm}
\setlength{\TPVertModule}{\TPHorizModule}
\textblockorigin{10mm}{10mm}
% start everything near the top-left corner
\setlength{\parindent}{0pt}

\usepackage{paralist}

\usepackage{fancyhdr}                     %Declares the package fancyhdr
\pagestyle{fancy}                         %Forces the page to use the fancy template
\fancyhead[L]{Exercises}
\fancyhead[R]{11.11.2011}
\fancyfoot[L]{\tiny{Version 0.3/10.11.2011}}
\fancyfoot[R]{\tiny{bho1@bfh.ch}}

%%
%%
\begin{document} 


% \begin{textblock}{300}(10, 20)
%  Sozialkompetenz
%  Methodenkompetenz
%  Selbstkompetenz
%  Fashkompetenz
% \end{textblock}

\center{{\Large \textsc{\underline{6502 Assembler}}}}
\setlength\parskip{0.5cm}
\begin{itemize}
\item[\textbf{Goals:}] Run first 6502 assembler program in c64 simulator. Edit code. Assemble Code. Read Opcode table. scan c64-kernal. Code 6502 assembler with different addressing modes
\item[\textbf{Knowledge assembled:}] Addressing modes, immediate adr. mode, absolute adr. mode, absolut with index adr. mode, implicit adr. mode
\item[\textbf{Exercises Intro:}] Work through \texttt{simple-hello-world}, \texttt{hello-world}, \texttt{count10} in this order.

\begin{enumerate}
\item Read, assemble, run program
\item Code assembler-code into 6502 binary code (by hand)
\item compare your code with the generated code
\end{enumerate}

\item[\textbf{Exercise First Steps:}] 
Modify programs to use other registers, less register, less code etc.
\item[\textbf{Exercise First Problem:}] Write a program which counts the character of a string \texttt{x} and prints ''\texttt{<x> has <number of x> character}''.
\item[\textbf{Exercise Addressing Modes:}] Try rewriting sum.asm by replacing the instructions using absolute with index addressing modes with other addressing modes
 \end{itemize} 

% \lstset{emph={%  Adjust any special keywords
%     printf%
%     },emphstyle={\color{red}\bfseries}%
% }%
\lstdefinelanguage{doxygen}%
{
    alsoletter={\\,[,],/,*,\,},%
    morekeywords=[2]{/**,*/,\\brief,%
                     \\LDX,\\STA,\\param[in],\\param[out]}
}
\lstdefinelanguage{6502} {
    otherkeywords = {:=},
    keywords = {asl, sta, inx, txa, cmp, bne, jmp, iny, inx, lda, rts, adc, sbc},
    keywordstyle=\textbf,
    commentstyle=\itshape
}
% \begin{lstlisting}%[language=[AVR]Assembler]
% ; void x(int myvar0, char* b) {}

% ; 16-bit addition
% .include "HelloWorld.asm"
% add r16, r0
% adc r17, r1
% \end{lstlisting}

\newpage

{\scriptsize
 \begin{lstlisting}[language=6502,escapechar=@]

; DISCO DISCO
; submitted by Anonymous
 
start:@\tikz[remember picture] \coordinate(startlabel);@
 inx
 txa
 sta $200, y
 sta $300, y
 sta $400, y
 sta $500, y
 iny
 tya
 cmp 16
 @\tikz[remember picture] \coordinate (bnedo);@bne do
 iny
 jmp start@\tikz[remember picture] \coordinate(startjump1);@
@\tikz[remember picture] \coordinate (do);@do:
 iny
 iny
 iny
 iny
jmp start@\tikz[baseline=0.5ex,remember picture] \coordinate (startjump2);@

 \end{lstlisting}%$
}

{\scriptsize
 \begin{lstlisting}[language=6502,escapechar=@]

                 * = 0600
0600   E8         INX
0601   8A         TXA
0602   99 00 02   STA $0200,Y
0605   99 00 03   STA $0300,Y
0608   99 00 04   STA $0400,Y
060B   99 00 05   STA $0500,Y
060E   C8         INY
060F   98         TYA
0610   C5 10      CMP $10
0612   D0 04      BNE $0618
0614   C8         INY
0615   4C 00 06   JMP $0600
0618   C8         INY
0619   C8         INY
061A   C8         INY
061B   C8         INY
061C   4C 00 06   JMP $0600
061F              .END
 \end{lstlisting}%$
}

\begin{figure}[h]
  \centering
\begin{minipage}[t]{.45\columnwidth} 
%  \includegraphics[scale=0.5]{/home/olivier/hti/ginf/pics/MOS6526.pdf}
\begin{tikzpicture}[remember picture,overlay] 
%\draw (startjump2) rectangle ++(.17cm,.3cm);
%\draw (startjump2) circle (0.1cm);
%\draw[step=.10cm,gray,very thin] (0,-2) grid (8,0); 
%\draw[step=1cm,gray] (0,-2) grid (8,0); 
%\node[circle,fill=blue!50,opacity=0.5] (n2) at (3,-0.30) {}; 
%\draw[red] (2.8,-0.1) rectangle (3.8, -0.5); 
%\draw [->,very thick,green,opacity=.5] (startjump2) node [right,text=red, opacity=1] {main method} to [out control=+(90:1)] (startlabel); 
%\path[->, red, thick] (startjump1) edge [bend left] (startlabel);
\draw[->, black, thick] (startjump2) .. controls +(7,-8) and +(15,1).. (startlabel);
\draw[->, black, thick] (startjump1) .. controls +(7,-5) and +(5,1).. (startlabel);

\draw[->, black, thick] (bnedo) .. controls +(-3,3) and +(-3,-3).. (do);
\end{tikzpicture}
\end{minipage}\hfill
  \caption{disassembler code Tabelle ab Adresse $0x0600$  }
  \label{fig:cia}

\end{figure}

\newpage
\begin{figure}[h]
  \centering
\begin{minipage}[t]{.45\columnwidth} 
%  \includegraphics[scale=0.5]{/home/olivier/hti/ginf/pics/MOS6526.pdf}
\end{minipage}\hfill
\begin{minipage}[t]{.45\columnwidth} 
{\scriptsize
\begin{tabular}{lllp{4cm}}
Pin 1 &GND &-- &GrouND: Masse (0V) \\
Pin 2-9 &PA0-PA7  &I/O  &Parallel port a signals. Bidirectional parallel port. \\
Pin 10-17 &PB0-PB7  &I/O &Parallel port b signals. Bidirectional parallel port. \\
Pin 18 &PC &O &Handshake output. A low pulse is generated after a read or write on port b. \\
Pin 19 &TOD &I &Time od day clock input. Programmable 50hz or 60hz. \\
Pin 20 &Vcc &--&Supply voltage: +5V DC \\
Pin 21 &$\overline{\mathrm{IRQ}}$ &O &Interrupt output to microprocessor input IRQ. \\
Pin 22 &R/$\overline{\mathrm{W}}$ &I &READ/WRITE input \\
Pin 23 &$\overline{\mathrm{CS}}$ &I &Chip select input. A low pulse will activate CIA. \\
Pin 24 &$\overline{\mathrm{Flag}}$ &I &Negative edge sensitive interrupt input. Can be used as a handshake line for either parallel port. \\
Pin 25 &{\o}2 &I &clock input connected to processor \\
Pin 26-33 &DB0-DB7  &I/O &Bidirectional data bus. Connects to processor data bus. \\
Pin 34 &$\overline{\mathrm{RES}}$ &I &Low active reset input. Initializes CIA. \\
Pin 35-38  &RS0-RS3  &I &Register select inputs. Used to select all internal registers for communications with the parallel ports, time of day clock and serial port (SP). \\
Pin 39 &SP &I/O &Serial Port bidirectional connection. An internal shift register converts microprocessor parallel data into serial data, and vice versa. \\
Pin 40 &CNT &I &Count input. Internal timers can count pulses applied to this input. Can be used for frequency dependant operations.

\end{tabular}
}
\end{minipage}
  \caption{Pin layout des CIA (Complex Interface Adapter) Chip für C64}
  \label{fig:cia}
\end{figure}

{\scriptsize
\begin{tabular}{|p{2cm}|c|c|c|c|c|c|c|c|c|}
\hline 
 & & CIA 1 Port B (\$DC01)  & Joy 2  & & & & & & \\
 \hline 
 & & PB7  & PB6  & PB5  & PB4  & PB3  & PB2  & PB1  & PB0  \\
 \hline 

 \begin{center}
 CIA1 Port A (\$DC00)   
 \end{center}
  & PA7  & STOP  & Q  & C=  & SPACE  & 2  & CTRL  & $<$-  & 1  \\
 \hline 
 PA6  & /  & \^{}  & =  & RSHIFT & HOME  & ;  & *  & \^A{\pounds}  & \\
 \hline 
 PA5  & ,  & @  & :  & .  & -  & L  & P  & +  & \\
 \hline 
 PA4  & N  & O  & K  & M  & 0  & J  & I  & 9  & Fire  \\
 \hline 
 PA3  & V  & U  & H  & B  & 8  & G  & Y  & 7  & Right  \\
 \hline 
 PA2  & X  & T  & F  & C  & 6  & D  & R  & 5  & Left  \\
 \hline 
 PA1  & LSHIFT & E  & S  & Z  & 4  & A  & W  & 3  & Down  \\
 \hline 
 PA0  &CRSR DN & F5  & F3  & F1  & F7  &CRSR RT  & RETURN  & DELETE  & Up  \\
 \hline 
 Joy 1  & & & & & Fire  & Right  & Left  & Down  & Up  \\
 \hline 

\end{tabular}
}
\end{document}



% \lstdefinelanguage[AVR]{Assembler}%
% {morekeywords={%
% % Arithmetic and Logic Instructions
% add,adc,adiw,sub,subi,sbc,sbci,sbiw,and,andi,or,or i,eor,
% com,neg,sbr,cbr,inc,dec,tst,clr,ser,mul,muls,mulsu ,fmul,
% fmulsu,des,%
% % Branch Instructions
% rjmp,ijmp,eijmp,jmp,rcall,icall,eicall,call,ret,re ti,cpse,
% cp,cpc,cpi,sbrc,sbrs,sbic,sbis,brbs,brbc,breq,brne ,brcs,
% brcc,brsh,brlo,brmi,brpl,brge,brlt,brhs,brhc,brts, brtc,
% brvs,brvc,brie,brid,%
% % Data Transfer Instructions
% mov,movw,ldi,lds,ld,ldd,sts,st,std,lpm,elpm,spm,in ,out,
% push,pop,xch,las,lac,lat,lsl,lsr,rol,ror,asr,swap, bset,
% bclr,sbi,cbi,bst,bld,sec,clc,sen,cln,sez,clz,sei,c li,
% ses,cls,sev,clv,set,clt,seh,clh,%
% % MCU Control Instructions
% break,nop,sleep,wdr},%
% morekeywords=[2]{%
% % AVR Assembler Directives
% .byte,.cseg,.db,.def,.device,.dseg,.dw,.endm,.endm acro,%
% .equ,.eseg,.exit,.include,.list,.listmac,.macro,.n olist,%
% .org,.set},%
% alsoletter={.,0,1,2,3,4,5,6,7,8,9},%
% alsodigit={x,X,b,B,\$},%
% morekeywords=[3]{%
% % AVR General Purpose Registers
% r0,r1,r2,r3,r4,r5,r6,r7,r8,r9,r10,r11,r12,r13,r14, %
% r15,r16,r17,r18,r19,r20,r21,r22,r23,r24,r25,r26,r2 7,%
% r28,r29,r30,r31},%
% sensitive=f,%
% morestring=[b]",%
% morestring=[b]',%
% morecomment=[l];%
% }[keywords,comments,strings]
% \lstloadlanguages{[AVR]Assembler}
% \lstset{%
% language=[AVR]Assembler,%
% frame=single,%
% breaklines=true,%
% basicstyle=\ttfamily,%
% commentstyle=\itshape\color{avrcom},%
% stringstyle=\color{avrstr},%
% keywordstyle=[1]\bfseries\color{avrcmd},%
% keywordstyle=[2]\bfseries\color{avrascmd},%
% keywordstyle=[3]\color{avrreg},%
% literate={,}{,}1}
% %}

% \lstdefinelanguage[6502]{Assembler}%
%   {morekeywords={%
%     % Arithmetic and Logic Instructions
%     asl,add,adc,adiw,sub,subi,sbc,sbci,sbiw,and,andi,or,ori,eor,
%     com,neg,sbr,cbr,inc,dec,tst,clr,ser,mul,muls,mulsu,fmul,
%     fmulsu,des,%
%     % Branch Instructions
%     rjmp,ijmp,eijmp,jmp,rcall,icall,eicall,call,ret,reti,cpse,
%     cp,cpc,cpi,sbrc,sbrs,sbic,sbis,brbs,brbc,breq,brne,brcs,
%     brcc,brsh,brlo,brmi,brpl,brge,brlt,brhs,brhc,brts,brtc,
%     brvs,brvc,brie,brid,%
%     % Data Transfer Instructions
%     sta,lda,ldi,lds,ld,ldd,sts,st,std,lpm,elpm,spm,in,out,
%     push,pop,xch,las,lac,lat,lsl,lsr,rol,ror,asr,swap,bset,
%     bclr,sbi,cbi,bst,bld,sec,clc,sen,cln,sez,clz,sei,cli,
%     ses,cls,sev,clv,set,clt,seh,clh,%
%     % MCU Control Instructions
%     break,nop,sleep,wdr},%
%     morekeywords=[2]{%
%     % AVR Assembler Directives
%     .byte,.cseg,.db,.def,.device,.dseg,.dw,.endm,.endmacro,%
%     .equ,.eseg,.exit,.include,.list,.listmac,.macro,.nolist,%
%     .org,.set},%
%     alsoletter={#,.,0,1,2,3,4,5,6,7,8,9},%
%     alsodigit={x,X,b,B,\$},%
%   morekeywords=[3]{%
%     % AVR General Purpose Registers
%     X,Y,r0,r1,r2,r3,r4,r5,r6,r7,r8,r9,r10,r11,r12,r13,r14,%
%     r15,r16,r17,r18,r19,r20,r21,r22,r23,r24,r25,r26,r27,%
%     r28,r29,r30,r31},%
%   sensitive=f,%
%   morestring=[b]",%
%   morestring=[b]',%
%   morecomment=[l];%
%   }[keywords,comments,strings]
% \lstloadlanguages{[6502]Assembler}


% \lstset{%
%   language=[6502]Assembler,
%   frame=single,
%   breaklines=true,
%   basicstyle=\ttfamily,
%   commentstyle=\itshape\color{avrcom},
%   stringstyle=\color{avrstr},
%   keywordstyle=[1]\bfseries\color{black},
%   keywordstyle=[2]\bfseries\color{green},
%   keywordstyle=[3]\color{avrreg},
%   literate={,}{,}1}


% lstset{
%     language=C,
%     basicstyle=\small\sffamily,
%     frame=tblr,
%     backgroundcolor=\color{yellow!20},
%     numbers=left,
%     xleftmargin=5.0ex,
%     numberstyle=\tiny,
%     numbersep=4pt,
%     stepnumber=2,
%     %showstringspaces=false,
%     keywordstyle=\color{blue}\bfseries
%     }
